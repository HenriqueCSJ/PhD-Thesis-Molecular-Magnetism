\chapter{Chapitre 1}

\section{Le couplage en transport : du direct à l'indirect}

Lorsque l'on souhaite sonder le magnétisme moléculaire par une mesure en courant, il faut que le "chemin" emprunté par les électrons soient très proche du centre magnétique. On peut, dans ce cadre, imaginer deux configurations : le couplage direct, où les électrons impliqués dans le transport jouent également un rôle direct dans le magnétisme de la molécule sondée; le couplage indirect, dans lequel les électrons impliqués dans le courant ne contribue pas au magnétisme de la molécule, mais le perturbe légèrement. Suivant la configuration adoptée, le mode de mesures sera différent, tout comme le sera l'impact sur le magnétisme. C'est ce que nous allons détailler maintenant.

\subsection{Le couplage direct}
Nous l'avons vu, le couplage direct implique que les électrons responsable du courant jouent également un rôle dans le magnétisme de la molécule. Cette dernière va osciller entre deux états de charge N/N+1 , chacun d'eux ayant sa propre configuration magnétique. L'analyse se fait en sondant la différence en énergie des différentes transitions N/N+1, le plus souvent par une technique de spectroscopie en tunnelling séquentiel. Le principal avantage de cette technique est qu'elle permet d'étudier différents états de charge~(nombre d'oxydation ou de réduction). En revanche, le caractère très invasif de la méthode ne permet pas d'espérer de long temps de vie pour les différents états du système. Cette méthode a été mise en œuvre expérimentalement dans [mettre les citations] avec des résultats mitigés, du fait notamment de la dégradation de la molécule lors de la fabrication du dispositif. Des études théoriques ont également été menée, permettant une analyse plus fine des résultats expérimentaux.

\subsection{Le couplage indirect}

Dans le cas du couplage indirect, les électrons responsables du courant ne participent qu'indirectement au magnétisme de la molécule. Le couplage à l'origine de cette interaction sera détaillé dans la suite. La mesure se fait par l'analyse statistique des modifications de conductance du système, la polarisation en tension source-drain et grille étant en général fixée~(par opposition à la spectroscopie en tunneling séquentiel). Dans ce cas, il n'y a pas de possibilité de modifier le nombre d'électron impliqués dans le magnétisme moléculaire. En revanche, la technique de mesure par couplage indirect se révèle beaucoup moins invasive. Cela garantie d'une part, la préservation des propriétés magnétique et d'autre part, l'observation de long temps de vie. Cette configuration a été utilisé dans deux dispositifs légèrement différents. Dans le premier, une deuxième molécule~(un nanotube) a été utilisé comme point quantique sonde, l'aimant moléculaire étant déposé sur sa surface [citation de Matias]. Dans le deuxième dispositif, le cœur magnétique étant fortement découplé des ligands périphériques, ces derniers ont joué le role de point quantique sonde[nous]. Quelques outils théoriques sont venu également faciliter l'interprétation des résultats [papier sur nanotube] mais également proposer de nouvelle expérience [article du les ligand et couple mécanique].

\section{Le couplage magnétique}
Dans la configuration directe présenté précédemment, le couplage entre le courant et le magnétisme est aisé à comprendre, les électrons participant au premier étant également directement impliqué dans le second. En revanche, dans la configuration indirecte, le couplage entre les ces deux domaines peut avoir plusieurs origines. Il a pour conséquence de rendre le potentiel chimique du point quantique dépendant du centre magnétique. Cette dépendance est fonction de la nature de l'interaction comme nous allons le montrer maintenant.

\subsection{Origine dipolaire}
Le couplage dipolaire est une interaction à distance entre deux moments magnétique. Chacun de ces moments génère un champ dipolaire qui va venir agir sur le second, et vice versa. La modification en énergie induite est fonction de la distance séparant les deux dipôles, ainsi que de leur orientation relative. Ceci s'exprime par:
\begin{eqnarray}
E = -\frac{\mu_0 \mu_B}{4\pi r^3}(3\mathbf{SnJn} - \mathbf{SJ}) \nonumber
\end{eqnarray}
où S et J sont les spins associés au deux moment magnétiques, r la distance qui les sépare et n la normale reliant les deux moments (cf figure??). Plusieurs remarques s'imposent. Premièrement, l'intensité de l'interaction est proportionnelle à l'inverse de la distance au cube. Elle devient très rapidement négligeable : pour un spin J=6, elle ne vaut plus que 10\,mT à 1\,nm. Deuxièmement, on peut imaginer deux configurations extrème : dans la situtation de la figure??, le couplage abouti à une organisation anti-ferromagnétique; dans celle présenté dans la figure ??, le couplage est au contraire ferromgnétique. Dans le cas de TbPc2, le plan des ligands est perpendiculaire à l'axe facile du moment magnétique. Cela correspond plutôt à la seconde configuration. De plus, la seule transition possible à basse température est $J_z=\pm6 \rightarrow J_z \mp 6$. Si l'on tient compte de ces remarques, la variation du potentiel chimique du point quantique sonde $\mu_{QD}$, est lié au renversement du moment magnétique par :
\begin{eqnarray}
\Delta \mu_{QD} = -\frac{\mu_0 \mu_B}{2\pi r^3}S_z\Delta J_z\nonumber
\end{eqnarray}

Celle-ci est directement proportionnelle à $\Delta J_z$. Autrement dit, le retournement du moment magnétique entraîne une variation de potentiel chimique constante.

\subsection{Couplage d'échange}
Le couplage d'échange est une interaction de contact entre deux moments magnétiques. Elle résulte d'un recouvrement des fonctions d'onde et peut favoriser deux situations opposés : si les spin s'alignent entre deux, elle est de type ferromagnétique; si l'orientation entre spin est opposée, elle est de type anti-ferromagnétique. Cette interaction s'exprime comme suit :
\begin{eqnarray}
E = A\mathbf{SJ} \nonumber
\end{eqnarray}
où A est la constante d'échange. Lorsque $A>0$, le couplage est anti-ferromagnétique, si $A<0$ il est ferromagnétique. La constante d'échange peu prendre des valeurs élevées en énergie : dans le cas du N@C$_{60}$ par exemple, la valeur de l'échange entre le spin de l'azote et les électrons du C$_{60}$ a été mesurée supérieure à 4\,T. Si l'on tient compte des même considération que dans le cas du couplage dipolaire, la modification d\^u à l'interaction d'échange qu’entraîne un retournement de l'aimantation peut s'exprimer de la façon suivante :
\begin{eqnarray}
\Delta \mu_{QD} = AS_z\Delta J_z\nonumber
\end{eqnarray}
Cette expression est semblable à celle obtenue pour le couplage dipolaire. Dans le cas où $A<0$ , les deux interactions produisent même des effets identiques. La principale différence se fait par l'intensité de l'interaction : si celle-ci n'est que de l'ordre du mT, elle est certainement dipolaire; si elle est en revanche de l'ordre de quelques dizaine de mT, l'interaction d'échange est l'interaction dominante.

\subsection{Le couplage magnéto-Coulomb}
L'origine de ce couplage est électrostatique. Si l'on considère un point quantique et un centre magnétique, cette interaction va coupler le potentiel chimique du premier à celui du second de telle sorte que :
\begin{eqnarray}
\Delta \mu_{QD} = C_{mc} \Delta \mu_{CM}
\end{eqnarray}
où $C_{mc}$ est la constate de couplage et $\mu_{CM}$ la variation du potentiel chimique du centre magnétique. Cette expression peut être simplifier, au regards des remarques précédentes, de la façon suivante :
\begin{eqnarray}
\Delta \mu_{QD} = C_{mc} g \mu_B  \Delta J_z B_z
\end{eqnarray}
Contrairement aux expressions précédentes, la variations du potentiel chimique associée à un retournement de l'aimantation n'est pas constante mais dépend du champ magnétique appliqué. Cela rend cette dernière facile à identifier dans nos mesures.

\section{Nature et intensité de l'interaction}
L'analyse de nos résultats passe par la détermination l'identification de l'interaction mise en jeu dans notre méthode de détection, à savoir, les sauts de conductance. Deux paramètres sont à évaluer pour clairement en identifier la nature : la dépendance en champ magnétique de la variation du potentiel chimique et l'intensité de l'interaction. Nous allons nous attacher à quantifier ces deux paramètres à l'aide de mesure en transport. La première étape sera consacré à l'étude de la hauteur des sauts de conductance. La seconde s'appuiera sur une étude de l'effet Kondo et sur l'influence de l'interaction sur ce dernier.

\subsection{Amplitude du saut de conductance}
Dans le chapitre théorique, nous avons montré que le potentiel chimique du point quantique était directement relié à la conductance différentielle mesurée. On peut résumer cette tendance par la relation suivante:
\begin{eqnarray}
\text{d}G = \frac{\partial G}{\partial \mu} \text{d} \mu
\end{eqnarray}
Si l'on se place dans une situation où $\frac{\partial G}{\partial \mu} = cst$, alors la variation observée en conductance sera une mesure directe de la variation du potentiel chimique. Si maintenant, on considère la variation du potentiel chimique en fonction du champ magnétique, sans renversement magnétique, on a $\text{d}\mu \propto \text{d}B$. Pour avoir $\frac{\partial G}{\partial \mu} = cst$, cela revient à se placer dans une zone ou $\frac{\partial G}{\partial B} = cst$. La Fig.??? montre une mesure de $G$ en fonction du champ magnétique $B$ et met en évidence la zone correspondante à $\frac{\partial G}{\partial B} = cst$. Dans celle-ci, un saut en conductance est directement proportionnel à la variation en potentiel chimique. Si cette dernière est dépendante en champ magnétique, la hauteur des sauts en conductance devrait également l’être. Or, la mesure présenté dans Fig.?? montre clairement que la hauteur des sauts en conductance ne varie pas en fonction du champ magnétique. Cette première observation nous permet d'éliminer l'interaction de type magnéto-Coulomb des mécanismes de coulage possible. On a donc à faire, soit à un couplage dipolaire, soit à un couplage d'échange. Seul l'analyse de l'intensité de l'interaction peut nous renseigner et c'est à sa détermination que nous allons nous attacher maintenant.

\subsection{Intensité de l'interaction}
L'intensité de l'interaction peut s'obtenir en comparant un système découplé de l'interaction avec le même système la subissant. Pour cela, on peut s'appuyer sur un phénomène universel tel l'effet Kondo de spin 1/2.

\subsubsection{L'effet Kondo 1/2}
La Fig??? présente la mesure d'un effet Kondo 1/2 en fonction du champ magnétique et de la tension source drain. A $B=0$, on observe un pic de conductance à $V_{ds}=0$. Lorsque l'on applique un champ magnétique, ce pic s'étale puis, puis se divise en deux pics de conductance. Cette séparation est directement induite par l'effet Zeeman. En extrapolant les maxima pour different B, on obtient une lecture de l'écartement Zeeman. En revanche, contrairement à ce que l'on pourrait attendre, les droites ne se croisent pas en $B=0$, mais en une valeur de champ fini $B_c$ supérieur à zéro. La valeur de $B_c$ est directement relier à la température Kondo $T_K$ par $k_bT_K = \mu_B B_c$. Cela traduit qu'il est nécessaire de fournie une énergie supérieur à celle de la température Kondo pour "casser" le singlet formé par le nuage Kondo et l'électron du point quantique. Regardons maintenant ce qu'il en est de notre système couplé.

\subsubsection{Effet Kondo du système couplé}
Si l'on effectue cette étude dans le cas du Kondo 1/2 couplé, on observe le même comportement général. Les pentes des droites extraites des extrema confirme qu'il s'agit d'un Kondo de spin 1/2. En revanche, la valeur de $B_c$ n'est plus positive mais négative. Tout se passe comme si le singlet était déjà "cassé" à champ magnétique nul. Cette première observation nous permet d'éliminer l'interaction d'échange anti-ferromagnétique. En effet, cette dernière aurait pour conséquences de décalé $B_c$ vers des valeur plus grande de champ magnétique. On a donc à faire, soit à une interaction dipolaire, soit à une interaction d'échange ferromagnétique.

Si on se place maintenant dans le cas où $T_k \sim 0$, l'extrapolation devrait donner $B_c=0$. Le décalage vers les valeurs négatives est uniquement du à l'interaction. Celle-ci a donc une intensité de l'ordre de plusieurs dizaines de milliKelvin~(il s'agit d'un valeur basse). Au regard des dimensions du système qui place le ligand à environs 1\,nm du centre magnétique, l'interaction dominante est l'interaction d'échange ferromagnétique.