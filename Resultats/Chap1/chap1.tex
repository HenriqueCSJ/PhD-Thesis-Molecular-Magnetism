\chapter{Chapitre 1}

\section{Le couplage en transport : du direct à l'indirect}

Lorsque l'on souhaite sonder le magnétisme moléculaire par une mesure en courant, il faut que le "chemin" emprunté par les électrons passent très proche du centre magnétique. On peut, dans ce cadre, imaginer deux configurations : le couplage direct où les électrons impliqués dans le transport jouent également un rôle direct dans le magnétisme de la molécule sondée; le couplage indirect dans lequel les électrons impliqués dans le courant ne contribue pas au magnétisme de la molécule, mais le perturbe légèrement. Suivant la configuration adoptée, le mode de mesures sera différent tout comme le sera l'impact sur le magnétisme. C'est ce que nous allons voir maintenant.

\subsection{Le couplage direct}
Nous l'avons vu, le couplage direct implique que les électrons responsable du courant jouent également un rôle dans le magnétisme de la molécule. Cette dernière va osciller entre deux états de charge N/N+1 , chacun d'entre-eux ayant se propre configuration magnétique. L'analyse se fait en sondant les différences en énergie des différentes transitions N/N+1, le plus souvent par une technique de spectroscopie en tunnelling séquentiel. Le principal avantage de cette technique est qu'elle permet d'étudier différents états de charge~(nombre d'oxydation ou de réduction). En revanche, le caractère très invasif de la méthode ne permet pas d'espérer de long temps de vie pour les différents états du système. Cette méthode a été mise en œuvre expérimentalement dans [mettre les citations] avec des résultats mitigés, du fait notamment de la dégradation de la molécule lors de la fabrication du dispositif. Des études théoriques ont également été menée, permettant une analyse plus fine des résultats expérimentaux.

\subsection{Le couplage indirect}

Dans le cas du couplage indirect, les électrons responsables du courant ne participent qu'indirectement au magnétisme de la molécule. Le couplage à l'origine de cette interaction sera détaillé dans la suite. La mesure se fait par l'analyse statistique des modifications de conductance du système, la polarisation du système en tension~(source-drain et grille) étant en général fixé~(par opposition à la spectroscopie en tunneling séquentiel). Dans ce cas, il n'y a pas de possibilité de modifier le nombre d'électron impliqués dans le magnétisme moléculaire. En revanche, la technique de mesure par couplage indirect se révèle beaucoup moins invasive. Cela garantie d'une part, la préservation des propriétés magnétique et d'autre part, l'observation de long temps de vie pour les états du système. Cette configuration a été utilisé dans deux dispositifs légèrement différents. Dans le premier, une deuxième molécule~(un nanotube) a été utilisé comme point quantique sonde, l'aimant moléculaire étant déposé sur sa surface [citation de Matias]. Dans le deuxième dispositif, le cœur magnétique étant fortement découplé des ligands périphériques, ces derniers ont joué le role de point quantique sonde[nous]. Quelques outils théoriques sont venu également faciliter l'interprétation des résultats [papier sur nanotube] mais également proposer de nouvelle expérience [article du les ligand et couple mécanique].