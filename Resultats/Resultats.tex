\chapter{Résultats expérimentaux}

\section{Le TbPc$_2$}
\subsection{Présentation}
\subsection{Origine du moment magnétique}
\subsection{Hamiltonien}
\subsection{Mesure de l'aimantation d'une assemblé}
\subsection{TbPc$_2$ et la spintronique}


\section{Mesure du retournement de l'aimantation}
\subsection{Transport à travers une boite quantique}
\subsection{Le couplage magnétisme-transport}
\subsubsection{Le couplage dipolaire}
\subsubsection{Le couplage d'échange}
\subsubsection{Le couplage magnéto-Coulomb}
\subsection{Le couplage mécanique}

\subsection{Intensité et nature de l'interaction}
\subsubsection{Amplitude des sauts de conductance}
\subsubsection{Intensité de l'interaction}

\subsection{Analyse des sauts en conductance}
\subsubsection{Méthode de détection}
\subsubsection{Interprétation physique de $\Delta g$}
\subsubsection{Choix du point de fonctionnement}
\subsubsection{Procédure d'alignement}

\subsection{Cycle d’hystérésis d'une assemblée versus une molécule isolé}
\subsubsection{A champ faible}
\subsubsection{A champ moyen}
\subsubsection{A champ fort}

\subsection{Dynamique du spin nucléaire}
\subsubsection{Temps de relaxation}
\subsubsection{Perturbations induites par la mesure}
\subsubsection{Influence de la tension de grille}
\subsubsection{Détermination de la température nucléaire}

\subsection{Perspectives}