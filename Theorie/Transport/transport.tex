

\chapter{Blocage de Coulomb}

Dans ce chapitre nous allons aborder le transport à travers une structure nanoscopique et nous allons nous intéresser en particulier au phénomène de blocage de Coulomb. Ce phénomène a d'abord été observé sur des échantillon métallique macroscopique composé de petit grain métallique par C.J. Gorter en 1951. En mesurant la conductance d'un de ces échantillon en fonction de la température, l'auteur a constaté une diminution inattendu à basse température. Cette diminution de la température à été attribué à l'aspect granuleux du métal. Plus précisément , pour ajouter un électron dans un des grains constituant le métal, il fallait fournir une énergie $e^2/C$ où $e$ est la charge de l'électron et $C$ est la capacité que l'on peut associé à un de ces grains. A haute température ($k_bT >> e^2/C$), cette énergie n'affecte pas la conductance du système. A basse température en revanche ($k_bT << e^2/C$), cette enérgie réduit l'habilité des électron à se mouvoir librement dans le métal ce qui entraîne une diminution de la conductance. Depuis, l'étude de ce phénomène à beaucoup évolué et de nos jours et le rôle central du grain de métal est souvent remplacé par des molécules, des point quantique de gaz d'électron à deux dimensions etc..

Dans ce chapitre, nous verrons tout d'abord qu'elles sont le différent paramètres physique d'un système type. En particulier, nous donnerons quelques conditions nécessaire à l'apparition d'un phénomène de blocage de Coulomb classique et quantique. Nous aborderons ensuite la notion de potentiel chimique et nous montrerons comment cette notion peut expliquer de façon intuitive le phénomène de blocage de Coulomb. Nous déterminerons notamment comment certain paramètre du système par une simple mesure de courant. Nous développerons ensuite un modèle plus quantitatif afin de déterminer le courant circulant à travers une structure nanoscopique. Pour cela, nous nous concentrerons sur le cas particulier d'une boite quantique oscillant entre deux états de charge.



\section{Les paramètres du système}
La description que nous proposons ici est celui d'un système à trois terminaux que l'on nommera la source, le drain et la grille, ainsi qu'un il\^ot dont la nature peut varier suivant les cas. On peut imaginer une molécule, un il\^ot métallique, un gaz d'électron deux dimensions etc.. L'il\^ot est couplé capacitivement au trois terminaux par trois capacitance ; $C_g$ pour la grille, $C_d$ pour le drain et $C_s$ pour la source. De plus, des barrière tunnel entre l'il\^ot et la source et l'il\^ot et le drain permettant le passage d'électron (sous certaines conditions). Ces deux barrières tunnels sont caractériser par les paramètre $\gamma_s$ pour le système source/il\^ot et $\gamma_d$ pour le système drain/il\^ot. La source et le drain sont considéré comme des matériaux massif métallique et dont les électrons obéisse à la statistique de Fermi-Dirac. Enfin, nous attribuons à l'il\^ot une taille caractéristique $L$. Tous ces paramètre sont représenté dans la Fig. \ref{description_systeme}. Nous allons maintenant regarder plus en détail chacun des ces paramètres.

\subsubsection{Les capacitances du système}
Comme expliqué précédemment, trois capacitances couples l'il\^ot central aux trois terminaux. De part ces capacitances, l'application d'un tension sur l'un ou plusieurs terminaux va modifier l'energie des électrons situés sur l'ilôt. Cette modification peut facilement s'exprimer comme suit :

\begin{eqnarray}
E = \frac{(C_sV_s + C_dV_d + C_gV_g)^2}{2(C_g + C_s + C_g)}=\frac{(C_sV_s + C_dV_d + C_gV_g)^2}{2C_{\Sigma}} \nonumber
\end{eqnarray}

De plus, ces capacitance vont induire un "co\^ut" énergétique à l'ajout d'un électron dans l'il\^ot central. L'ajout d'un électron est associé à l'énergie $\frac{E_c}{2}$~(nous verrons l'utilité du facteur un-demi dans la suite). Cette énergie est donné par :
\begin{eqnarray}
\frac{E_c}{2} = \frac{e^2}{2(C_s+C_d+C_g)}=\frac{e^2}{2C_{\Sigma}} \nonumber
\end{eqnarray}


Ces cette énergie qui est à l'origine de la diminution de la conductance observée par C.J. Gorter en 1951. On voit ici une première condition nécessaire à l'apparition du phénomène du blocage de Coulomb : $E_c >> k_bT$.

On peut donc exprimer l'énergie d'un il\^ot contenant N électrons et soumis à trois tensions $V_g$, $V_d$ et $V_s$ comme suit :
\begin{eqnarray}
U(N) = \frac{1}{2C_{\Sigma}} (-|e|N + C_sV_s + C_dV_d + C_gV_g)^2
\end{eqnarray}

\begin{figure}
\caption{Paramètres caractérisant un système à trois terminaux}
\label{description_systeme}
\end{figure}



\subsubsection{La taille de l'il\^ot}
Comme nous l'avons précisé précédemment, nous étudions des structure nanoscopique ou autrement dit $L\sim nm$. Pour de telle dimension, on observe une quantification des différent état du système il\^ot. Les spectre énergétique de l'il\^ot peut s'exprimer en fonction de trois nombre quantique $n_x$,$n_y$ et $n_z$ à travers la relation suivante:
\begin{eqnarray}
E_n = \frac{\pi^2 \hbar^2}{2m}(\frac{n_x^2}{L_x^2} + \frac{n_y^2}{L_y^2} + \frac{n_z^2}{L_z^2}) \nonumber
\end{eqnarray}


Il s'agit bien entendu ici d'une expression très simplifié. Dans le cas de molécules par exemple, cette quantification simpliste est remplacé par la quantification correspondante aux orbitale moléculaire. Pour résoudre expérimentalement cette quantifications des niveaux la condition $\frac{\pi^2 \hbar^2}{2mL} >> k_bT$. Le seul moyen d'action de l'expérimentateur se fait à la fois sur $L$ en diminuant la taille des objets observés et $T$ en utilisant des frigo à dilution. Lorsque cette condition est rempli, on rentre dans le régime du blocage de coulomb quantique.
\newline


\subsubsection{Les paramètre de couplage tunnel $\gamma_{s/d}$}
On peut voir ces coefficient comme définissannt la "facilité" avec laquelle les électrons peuvent passer par effect tunnel de la source ou du drain vers l'il\^ot et vice-versas. Les valeur $\gamma_{s/d}$ sont déterminante dans la valeur du courant qui va \^etre mesuré dans notre système. De plus, le couplage à la source et au drain contribue à l'élargissement des niveaux d'énergie d'une largeur $\Delta E$ donnée par :
\begin{eqnarray}
\Delta E_{\text{intrinsèque}} = h (\gamma_s + \gamma_d)
\end{eqnarray}
Cette élargissement est appelé élargissement intrinsèque par opposition à l'élargissement induit par la température. On peut deviner ici une seconde condition nécéssaire à l'apparition du phénomène de blocage de Coulomb à savoir $\Delta E_{\text{intrinsèque}} << E_c$. De plus, dans un régime de blocage fort on a $\Delta E_{\text{intrinsèque}} << k_bT$. Si cette dernière condition est remplie, on peut donc avoir accès par l'intermédiaire de des distribution de Fermi-Dirac à la température du système. C'est cette propriété qui est utilisé dans le thermomètre à blocage de Coulomb.





\section{La notion de potentiel chimique}
La notion de potentiel chimique est à mes yeux une des notions les plus importantes afin de comprendre de manière simple et intuitive le phénomène de blocage de Coulomb. Un exemple de son utilisation dans la cadre du transport quantique peut \^etre trouvé dans la très belle et très pédagogique revue de Hanson \textit{et Al.}. Dans cette section, nous allons tout d'abord présenté le concept de potentiel chimique. Nous exprimerons ensuite, à partir des considération exposé dans la partie précédente, le potentiel chimique de la source, du drain et surtout de l'ilôt central.

\subsubsection{Définition}

On recontre souvent le potentiel chimique en thermodynamique lorsque l'on s'intéresse aux systèmes ouvert échangeant des particules. Cette grandeur défini la variation d'énergie d'un système du à la modification du nombre de particule qui le compose. On le trouve parfois défini comme suit :
\begin{eqnarray}
\mu = \frac{\partial U}{\partial N} \nonumber
\end{eqnarray}
$U$ étant l'énergie du système et $N$ le nombre de particule. Dans la suite, nous allons plut\^ot adopter la notation de Hanson et Al. et prendre la définition suivante :
\begin{eqnarray}
\mu(N) = U(N) - U(N-1)
\end{eqnarray}
ou $\mu(N)$ est la modification en énergie apporté par l'ajout de la $N^\text{nième}$ particules, $U(N)$ et $U(N-1)$ étant respectivement l'énergie du système avec $N$ et $N-1$ particules.

Maintenant, partant de cette définition on peut imaginer un système de trois réservoirs. Supposons maintenant que l'on puisse atribuer à chacun des ces réservoirs un potentiel chimique et que deux plus, ces potentiels chimiques soient alignés. Je nommerai ces potentiel chimique $\mu_{droit}$, $\mu_{centre}$ et $\mu_{gauche}$. On a donc $\mu_{droit}=\mu_{centre}=\mu_{gauche}$.

Si maintenant je prends une particule dans le réservoir de droite et que je là mets au centre. L'énergie du réservoir de droite va varié de $-\mu_{droit}$ tandis que l'énergie du réservoir du centre va varier de $\mu_{centre}$. En revanche la variation totale en énergie du système est nulle. 

En faisant de m\^eme avec le réservoir de gauche, on arrive à la conclusion que les trois configuration ; la particule à droite au centre ou à gauche ont la m\^eme énergie ou autrement dit, sont dégénéré. Nous verrons dans la suite que une telle configuration sera appelé point de dégénérescence. 

De fait, la particule est libre de circuler d'un point du système à l'autre. Nous voyons donc que la notion de potentiel chimique nous permet d'identifier ou non la possibilité pour des particules de passer d'un réservoir à l'autre en fonction du potentiel chimique qui leur est associés. Nous utiliserons cette propriété dans la suite pour définir les contitions de circulation d'électron dans notre système. Mais pour cela, il faut définir ce qu'est le potentiel chimique dans le cas de nos trois réservoir à savoir: la source, le drain et bien sûr l'ilôt.


\subsubsection{Les potentiels chimique de la source et du drain.}
L'expression du potentiel chimique de la source et du drain est directement donnée par $\mu_i = e V_i$ ou $i=source/drain$. Il s'agit en fait du niveau de Fermi des électron dans la source et le drain (à ne pas confondre avec l'énergie de Fermi). Si l'on veut maintenant savoir quel est la probabilité dans un métal de niveau de fermi $\mu_F$ de trouvé un électron de potentiel chimique $\mu$, je peux utiliser la distribution de fermi et cette probabilité est donc égale à :
\begin{eqnarray}
p(\mu) = \frac{1}{1 + \exp{(\frac{\mu - \mu_F}{k_bT})}} \nonumber
\end{eqnarray}
 On obtient donc en fonction des tensions source et drain:
\begin{eqnarray}
p_i(\mu) = \frac{1}{1 + \exp{(\frac{\mu - eV_i}{k_bT})}}
\end{eqnarray}
ou $i=source/drain$. Nous verrons dans la suite que cette notion est essentielle dans la détermination du courant qui traverse notre structure.

\begin{figure}
\caption{Distribution de Fermi pour une tension $V_i$ donné}
\label{distrib_fermi}
\end{figure}



\subsubsection{Le potentiel chimique de l'il\^ot}
Si l'expression du potentiel chimique de la source et du drain n'a rien de compliqué (dans le cas d'életrode normale tout du moins), on ne peut pas en dire de m\^eme de celle de l'il\^ot. C'est m\^eme là que réside toute la difficulté de la compréhension d'une expérience. Heureusement, dans la partie précédente nous avons déjà fait le bilan des différentes énergie en jeux dans le système. A savoir, nous devons prendre en compte l'énergie électrostatique du système, l'énergie d'interaction électron-électron ainsi que la discrétisation des niveaux d'énergie dans l'il\^ot. Tout ceci donne :
\begin{eqnarray}
U(N) = \underbrace{\frac{1}{2C_{\Sigma}} (-|e|N + C_sV_s + C_dV_d + C_gV_g)^2}_{\text{couplage électrostatique et énergie de charge}}
+ 
\underbrace{\sum_{n=1}^{N} E_n}_{\substack{\text{énergie liés aux} \\\text{aux états discret}}}
\end{eqnarray}
On peut également tenir compte d'un éventuel champ magnétique en faisant le remplacement suivant :
\begin{eqnarray}
\sum_{n=1}^N E_n = \sum_{n=1}^N E_n(B) \nonumber
\end{eqnarray}
c'est à dire en attribuant à chaque niveau discret, une dépendance en champ magnétique. Nous verrons rapidement dans la suite comment cela se traduit dans le cas d'un système simple. 

Une fois l'énergie en fonction de $N$ exprimer simplement, il suffit d'appliquer la définition précédente à savoir :
\begin{eqnarray}
\mu(N) = U(N) - U(N-1) \nonumber
\end{eqnarray}
On se retrouve avec une expression relativement simple du potentiel chimique :
\begin{eqnarray}
\mu(N) = (N-\frac{1}{2})\frac{e^2}{C_{\Sigma}}
+ 
\frac{e}{C_{\Sigma}}(C_gV_g + C_sV_s + C_dV_d)
+
E_N(B)
\end{eqnarray}

Si maintenant, on se rappelle que nous avons défini plus haut l'énergie de charge $\frac{E_c}{2} = \frac{e^2}{C_{\Sigma}}$, nous pouvons réécrire la relation précédente sous la forme :

\begin{eqnarray}
\mu(N) = (N-\frac{1}{2})E_c
- 
\frac{E_c}{|e|}(C_gV_g + C_sV_s + C_dV_d)
+
E_N(B)
\label{pot_chim}
\end{eqnarray}
L'énergie $E_c$ est donc la quantité d'énergie d\^u à la répulsion Coulombienne qui sépare deux potentiels chimiques d'état de charge différents.


\section{Détermination des conditions de circulation d'un courant}
Pour rendre l'exposé qui va suivre plus clair, nous allons le décomposé en trois partie. Dans la première partie, nous allons voir quelles sont les conditions à remplir pour qu'un électron du drain puisse aller dans l'il\^ot. Dans la deuxième partie, nous ferrons de m\^eme pour la source. Enfin, dans la dernière partie, nous exploiterons les résultats obtenues pour en déduire les conditions pour qu'un courant circule dans notre structure ainsi que le signe de ce courant en fonction des paramètres. Afin d'adapter les solutions trouvées au condition expérimentale, on posera $V_s = 0$ car dans la grande majorité des dispositif, une des électrodes est directement connecter à la masse. Ce qui donnera $V_d=V_{ds}$, $V_{ds}$ étant la tension appliqué à l'échantillon au traver de la source et du drain.

\subsubsection{Charge de l'il\^ot par le drain}
Comme nous en avons discuté précédemment, pour qu'un particule (ici un électron) puisse passer d'un réservoir à l'autre, il faut que son potentiel chimique soit identique dans les deux réservoirs. Si l'on adapte se raisonnement à notre système, il faut donc qu'il y ait dans le drain des électrons dont le potentiel chimique corresponde à celui de cette électron une fois sur l'il\^ot. Supposoent donc l'il\^ot dans l'état de charge $N-1$, pour passer à l'état de charge $N$, il faut qu'il y ait au moins un électron dont le potentiel chimique soit égale à $\mu(N)$. Il nous suffit d'oberver la courbe de la Fig. \ref{distrib_fermi} pour comprendre que cela suppose :
\begin{eqnarray}
-|e|V_{ds} \geq \mu(N)
\end{eqnarray}
Ce qui conduit à la relation suivante :
\begin{eqnarray}
-|e|V_{ds} \geq (N-\frac{1}{2})\frac{e^2}{C_{\Sigma}}
-
\frac{|e|}{C_{\Sigma}}(C_gV_g + C_sV_s + C_dV_d)
+
E_N(B) \nonumber
\end{eqnarray}
En se rappelant que $V_s= 0$ et que $V_{ds} = V_d$, on en déduit la relation suivante :
\begin{eqnarray}
V_{ds} \leq \frac{1}{C_{\Sigma}}(C_gV_g + C_dV_{ds}) + \frac{E_N(B)}{|e|} - (N-\frac{1}{2})\frac{|e|}{C_{\Sigma}} \nonumber
\end{eqnarray}
Cette relation peut se réécrire de la façon suivante :
\begin{eqnarray}
V_{ds} \leq \frac{1}{C_g + C_s} \{C_gV_g - \frac{C_{\Sigma}}{|e|}[E_N(B) + (N-\frac{1}{2})E_c] \}
\end{eqnarray}
De cette formule nous pouvons en déduire deux relations importantes. Comme on peut le voir, l'état de charge $N$ décale la droite caractéristique de charge de l'il\^ot. De cette constatation, on peut en déduire que la séparation en tension de grille $\Delta V_g$ et liée à l'énergie du système par :
\begin{eqnarray}
\frac{C_g}{C_{\Sigma}} |e| \Delta V_g = E_c + \Delta E
\end{eqnarray}
Le deuxième enseignement est que la zone de transition entre charge et décharge dans le plan ($V_g$,$V_{ds}$) est délimité par une droite dont la pente est donnée par les différentes capacitances du système. Cette pente est donnée par 
\begin{eqnarray}
\frac{C_g}{C_g + C_s}
\end{eqnarray}



\begin{figure}
\caption{Représentation de la charge et de la décharge de l'il\^ot dans le plan ($V_g$,$V_{ds}$)}
\label{charge_discharge}
\end{figure}



\subsubsection{Charge de l'il\^ot par la source}
Un raisonnement similaire au précédent et en se rappelant que $V_s = 0$ conduit à la relation suivante :

\begin{eqnarray}
V_{ds} \geq -\frac{1}{C_d} \{C_gV_g + \frac{C_{\Sigma}}{|e|}[E_N(B) + (N-\frac{1}{2})E_c] \}
\end{eqnarray}


On peut extraire une deuxième pente qui correspond à la charge ou la décharge de l'il\^ot par la source.


\subsubsection{Condition de circulation du courant}

Si l'on reprend les deux paragraphes précédent, on peut imaginer quatre situtation :
\begin{enumerate}
\item aucun électron ne peut \^etre chargé ni par la source ni par le drain. L'état de charge reste à N.
\item un électron peut \^etre chargé à la fois par la source et par le drain. Il va donc y demerer et l'état de charge est N+1.
\item un électron ne peut \^etre charge que par la source. Dans ce cas, il finit par se décharger dans le drain
\item un électron ne peut \^etre chargé que par le drain. Dans ce cas, il finit par se décharger dans la source.
\end{enumerate}
Dans les cas un et deux, l'état de charge de l'il\^ot est bien défini et on se trouve dans le régime de blocage de Coulomb. Dans le cas 3 les électrons circulent de la source vers le drain. Un courant positif est donc mesuré. Dans le cas 4, les électrons circulent du drain vers la source. Un courant négatif est donc mesuré. L'emsemble de ces régimes est représenté dans la Fig. \ref{charge_discharge}.








\section{\'Equation pilote}
Une manière plus précise de poser le problème du courant est d'utiliser la méthode des équation pilote. On peut diviser la procedure à suivre en trois étapes. La première étape consiste à déterminer les différents état possible de l'ilôt, de leur attribuer à chacun une probabilité et de mettre en évidence ensuite les différentes relations de transition d'un état à l'autre. Une fois ceci fait, il faut déterminer les paramètre physique permetant le passage d'un état à l'autre en exprimant les taux de transition que l'on notera $\Gamma$. Enfin, à partir des taux de transitions et des différentes probabilité d'état, on peut exprimer le courant circulant dans le système.

\subsubsection{Les différents états du système et leur probabilités}
On suppose ici que l'état de charge de l'ilôt est soit $N=0$ soit $N=1$. Si l'on tient compte de l'état de spin de l'électron, l'on peut associé à chacun de ces états une probabilité (P$_0$,P$_-$ et P$_+$ respectivement) en attribuant un $+$ pour l'état spin up et $-$ à l'état de spin down. On peut facilement établir entre ces probabilité les relations suivantes:

\begin{eqnarray}
\frac{dP_0}{dt} &=& \Gamma_{+ \rightarrow 0}P_+ + \Gamma_{- \rightarrow 0}P_-  -(\Gamma_{0 \rightarrow +}P_0 + \Gamma_{0 \rightarrow -}P_0) \nonumber \\
\frac{dP_\pm}{dt} &=& \Gamma_{0 \rightarrow \pm}P_0 - \Gamma_{\pm \rightarrow 0}P_\pm \nonumber
\end{eqnarray}
ou $\Gamma_{\alpha \rightarrow \beta}$ est le taux de transition de l'état $\alpha$ à l'état $\beta$. 

En régime permanent, les differentes probabilités ne dépendent plus du temps et on peut donc en déduire les relations suivantes :
\begin{eqnarray}
P_0 &=& \frac{\Gamma_{+ \rightarrow 0}P_{+} + \Gamma_{- \rightarrow 0}P_{-}}{\Gamma_{0 \rightarrow -} + \Gamma_{0 \rightarrow +} }\\
P_{\pm} &=& \frac{\Gamma_{\pm \rightarrow 0}}{\Gamma_{0 \rightarrow \pm}}P_0 
\end{eqnarray}

Ce qui peut être reformuler de la façon suivante:
\begin{eqnarray}
P_0 &=& \frac{1}{1 + \frac{\Gamma_{0 \rightarrow +}}{\Gamma_{+ \rightarrow 0}} + \frac{\Gamma_{0 \rightarrow -}}{\Gamma_{- \rightarrow 0}}} \\
P_{\pm} &=& \frac{\Gamma_{\pm \rightarrow 0}}{\Gamma_{0 \rightarrow \pm}}P_0 
\end{eqnarray}


Il nous faut maintenant exprimer les taux de transition $\Gamma_{0 \rightarrow \pm}$ et $\Gamma_{\pm \rightarrow 0}$ en fonction des paramètres du système. 

\subsubsection{Détermination des taux de transfert}
Nous allons tout d'abord nous intéresser au taux de transfert $\Gamma_{0 \rightarrow \pm}$ car le raisonement à faire est très proche de celui effectué précédemment dans le cadre des potentiels chimiques. 

Comme nous l'avons déjà montré plus haut, il y a deux façon de charge l'ilôt : par la source ou par le drain. On a donc :
\begin{eqnarray}
\Gamma_{0 \rightarrow \pm} = \Gamma_{0 \rightarrow \pm}^s + \Gamma_{0 \rightarrow \pm}^d
\end{eqnarray}
où nous avons diviser le taux de transfert en un taux de transfert source et un taux de transfert drain. Il s'agit donc de trouver dans la source (ou le drain) un électron dont le potentiel chimique correspond à la transition $0\rightarrow \pm$. Nous noterons le potentiel chimique associé à cette transition $\mu_{\pm}$ dans la suite. 

Son expression peut facilement se déduire de l'Equ. \ref{pot_chim} et exprimer sous la forme suivante :
\begin{eqnarray}
\mu_{\pm} = \frac{1}{2}E_c - \frac{E_c}{|e|}(C_gV_g + C_sV_s + C_dV_d)~ \pm \underbrace{ \frac{1}{2}g \mu_B B}_{\text{terme Zeeman}}
\end{eqnarray}
ou  $\mu_B$ et le magnéton de Bohr, $g$ est le facteur de Landé et $B$ est le champ magnétique appliqué au système.

Si on se réfère au paragraphes précédent, le probabilité de trouver un électron dans la source ou dans le drain dont le potentiel chimique est égal à $\mu_{\pm}$ est donnée par :
\begin{eqnarray}
p_i(\mu_\pm) = \frac{1}{1 + \exp{(\frac{\mu_\pm - eV_i}{k_bT})}}
\end{eqnarray}
ou $i=source/drain$. 

Du fait de la présence d'une barrière tunnel entre la source ou le drain et l'ilôt, cette probabilité doit être pondéré par un terme relatif au couplage que l'on notera $\gamma_i$ ou $i=source/drain$.
On peut donc écrire:
\begin{eqnarray}
\Gamma_{0 \rightarrow \pm} =& \Gamma_{0 \rightarrow \pm}^s + \Gamma_{0 \rightarrow \pm}^d  \nonumber \\
 =& \gamma_s p_s(\mu_\pm) + \gamma_d p_d(\mu_\pm)
\end{eqnarray}

Par un raisonnement identique on peut déterminer $\Gamma_{\pm \rightarrow 0}$. La seule modification au raisonnement est qu'il faut cette fois-ci qu'un état soit libre dans la source ou dans le drain ce qui correspond à une probabilité de :
\begin{eqnarray}
1 - p_i(\mu_\pm) &=& 1 - \frac{1}{1 + \exp{(\frac{\mu_\pm - eV_i}{k_bT})}} \nonumber \\
 &=& \frac{\exp{(\frac{\mu_\pm - eV_i}{k_bT})}}{1 + \exp{(\frac{\mu_\pm - eV_i}{k_bT})}}
\end{eqnarray}
ou $i=source/drain$.
En effectuant cette subtitution, on trouve facilement :
\begin{eqnarray}
\Gamma_{\pm \rightarrow 0} =& \Gamma_{\pm \rightarrow 0}^s + \Gamma_{\pm \rightarrow 0}^d  \nonumber \\
 =& \gamma_s \{1 - p_s(\mu_\pm)\} + \gamma_d \{1-p_d(\mu_\pm)\}
\end{eqnarray}
Nous avons désormais tous les éléments pour exprimer le courant circulant dans notre système. C'est cette dernière étape que nous alons aborder maintenant.
\subsubsection{Détermination du courant}
De part la loi de conservation, on peut calculer indifféremment le courant au niveau de la source ou au niveau du drain. Si l'on se place du c\^oté de la source, on peut voir que le courant est composé d'un composante positive de part les électrons qui quittent la source pour l'il\^ot, et une composante négative  de part les électrons de l'il\^ot qui se décharge dans la source. Ces deux composantes peuvent s'exprimer de la façon suivante:
\begin{eqnarray}
I = |e| \gamma_s [(\Gamma_{0 \rightarrow +}^s + \Gamma_{0 \rightarrow -}^s) P_0 - \{ \Gamma_{+ \rightarrow 0}^s P_{+} + \Gamma_{- \rightarrow 0}^s P_{-}  \}]
\end{eqnarray}
On peut voir dans la Fig. \ref{SimulatedCoulombMap} le courant correspondant ainsi que ça dérivé relative à la tension $V_{ds}$ dans le plan ($V_g$,$V_{ds}$). Le tracé a été fait sans et avec champs magnétique. Dans le deuxième cas, on voit clairement apparaitre des ligne supplémentaire correspondant à un état excité généré par la levé de dégénerescence par effet Zeeman.
\begin{figure}
\caption{Courant correspant à un état de charge (0,1) dans le plan ($V_g$,$V_{ds}$) sans champ magnétique et avec champ magnétique}
\label{SimulatedCoulombMap}
\end{figure}


\subsubsection{Quelques remarques}
Il convient toutefois de faire quelques remarque sur cette méthode des équation pilote. Tout d'abord, nous n'avons pas tenu compte des relaxation à l'intérieur de l'il\^ot. Pour inclure de tel phénomène, il faudrait introduire des taux de transfert du type $\Gamma_{\pm \rightarrow \mp}$. De plus, afin que dans le cas d'un système isoler le système retrouve une distibution de type Bolzman, il faudra s'assurer que ces deux taux de transvers obéisse à la relation suivante :
\begin{eqnarray}
\frac{\Gamma_{+ \rightarrow -}}{\Gamma_{- \rightarrow +}} = \exp(\frac{\mu_{+}- \mu_{-}}{k_bT}) \nonumber
\end{eqnarray}

Deuxième remarque, l'élargissement des niveau au sein de l'il\^ot n'est pas pris en compte. Dans le régime fortement bloqué que nous avons choisi comme modèle ici, cette hypothèse est raisonnable. 

Troisième remarque, la méthode de l'équation pilote n'est valable que dans le cas d'évènement de tunnel des électron indépendant les uns de autres. Cette méthode ne peut donc pas \^etre utiliser dans le traitement du cotunneling que l'on verra plus loin.
