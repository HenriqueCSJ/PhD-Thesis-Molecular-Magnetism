\chapter*{Conclusion et perspectives}
\markboth{\MakeUppercase{Conclusion et perspectives}}{\MakeUppercase{Conclusion et perspectives}}
\addcontentsline{toc}{chapter}{Conclusion et perspectives}
\setcounter{figure}{0}

Les travaux que nous venons de présenter ne sont que les premières étapes dans le développement d'une véritable spintronique à l'échelle de la molécule unique. Ils ont cependant révélé de nombreux aspects de l'interaction entre transport électronique et magnétisme moléculaire.

Ils démontrent tout d'abord la faisabilité d'une électronique moléculaire dans laquelle le magnétisme moléculaire peut être sondé. En effet, nous avons pu montrer que la conductance de notre transistor était dépendante de l'orientation du moment magnétique d'une molécule unique. Correctement utilisée, cette propriété permettrait de lire une information codée dans l'orientation de ce moment. On peut imaginer une mémoire entièrement constituée de jonctions moléculaires, et donc une grande densité de stockage.

Cette capacité de détection fait également de notre système, un détecteur d'une grande sensibilité, qui pourrait être utilisé dans le cadre du magnétisme moléculaire, pour l'investigation des propriétés quantiques des aimants moléculaires. La méthode que nous avons développé afin de reconstituer le cycle d'aimantation à l'échelle d'une molécule unique pourrait contribuer à cette étude. Notamment, l'influence de l'interaction entre l'environnement et le magnétisme moléculaire mériterait une attention particulière, même si nos résultats apportent une première information en montrant que cette dernière peut induire des transitions résonantes supplémentaires.

En outre, nous avons présenté une nouvelle technique basée sur le phénomène de QTM, permettant de mesurer, de manière non destructive, l'état d'un spin nucléaire unique, ce qui porte la sensibilité de notre système à quelques millièmes de $\mu_B$. A l'aide de celui-ci, nous avons étudié la dynamique du spin nucléaire de terbium. En particulier, nous avons mis en évidence le long temps de vie des états de spin nucléaire. La dépendance de cette dynamique vis-à-vis de l'environnement électrostatique a également été démontré même si des études plus précises seront nécessaire afin d'identifier la ou les mécanismes responsables de cette dépendance.

Cependant, de nombreux aspects de nos travaux peuvent encore être améliorés. La fabrication de nos échantillon tout d'abord qui, à ce stade, n'exploite pas encore les possibilités offertes par l'approche ``\textit{Bottom-Up}" évoquée dans le premier chapitre. En collaboration avec notre groupe, Sébastien Liatar, dans le cadre de sa thèse, a mené une étude préliminaire allant dans cette direction. La technique sur laquelle il travaillait consistait à venir attaché une molécule, par des moyens chimiques, à deux billes d'or. Il faisait ensuite, à partir de ces billes, croître des battonnet d'or de plusieurs centaines de nanomètre de longueur qu'il serait ensuite facile de contacter. Ces travaux n'ont malheureusement pas pu être menés à leur terme, et n'ont donc pas conduit à la fabrication d'un dispositif fonctionnel. Il sera donc nécessaire d'améliorer cette technique afin d'obtenir des dispositif plus aisément mais surtout, de manière reproductible.

Outre l'aspect fabrication, une bonne compréhension des phénomènes de couplage entre magnétisme moléculaire et environnement nous parait indispensable. Il faut pour cela parvenir à comprendre quels sont les mécanismes à l'origine des transitions induites que nous avons pu mesurer. Une première étude de leur position en fonction du champ transverse nous permet de penser que ces dernières sont dues à l'interaction entre la molécule aimant et un second système magnétique. Mais la nature de ce système magnétique et de l’interaction qui le couple à l'aimant moléculaire restent encore à déterminer. La travail des théoriciens à ce sujet nous apparait indispensable à la poursuite des mesures mais surtout à l'analyse de ces dernières.

De même, si le rôle important de l'environnement électrostatique, et donc du régime de transport, sur le magnétisme de notre aimant moléculaire est indiscutable, le ou les mécanismes régissant cette dépendance ne sont pas encore identifiés. Ils conduisent pourtant à une modification de la vitesse de relaxation~(cf Chap. 3), mais aussi à une suppression des transitions induites lorsque l'on se place en régime Kondo. Il reste à démontrer que ces différent phénomènes ne sont que la manifestation d'un seul et m\^eme mécanisme physique, ou bien au contraire, qu'ils sont liés à différents interactions.