\chapter*{Conclusions et Perspectives}

Les travaux que nous venons de présenter ne sont que les premières étapes dans le développement d'un véritable spintronique à l'échelle de la molécule unique. Ils ont cependant révéler de nombreux aspects de l'interaction qu'il peut y avoir entre transport électronique et magnétisme moléculaire.

Ils démontrent tout d'abord la faisabilité d'une électronique moléculaire dans laquelle le magnétisme moléculaire peut être sondé et utilisé afin de stocker une information. En effet, nous avons pu montrer que la conductance de notre transistor était dépendante de l'orientation du moment magnétique d'une molécule unique. Correctement utilisé, cette propriété pourrait être utilisé pour lire une information codée dans l'orientation de ce moment. 

Cette capacité de détection fait également de notre système, un détecteur d'une grande sensibilité, qui pourrait être utilisé dans le cadre du magnétisme moléculaire, pour l'investigation des propriétés quantiques des aimants moléculaires. La méthode que nous avons développé afin de reconstituer le cycle d'aimantation à l'échelle d'une molécule unique pourrait être utilisé à cette fin. L'étude de l'influence des interactions de l'environnement sur le magnétisme moléculaire pourrait trouvé dans ces mesures une nouvelle source d'information. Les résultats que nous avons présenté apporte une première réponse en montrant que des transitions résonante supplémentaire pouvait être induite par l'interaction avec l'environnement. L'analyse de ce phénomène, de façon détailler reste encore à faire.

De plus, nous avons présenté une technique basé que le phénomène de QTM permettant de mesurer de façon non destructive un spin nucléaire unique. Cette technique a notamment été utilisée afin d'étudier la dynamique de ce dernier. Le long temps de vie des spin nucléaire a été mis en évidence. En outre, nous avons démontré que la dynamique du spin nucléaire pouvait être affecté par l’environnement électro