\section*{Remerciements}

On m'avait prévenu : « tu finis vraiment ta thèse quand tu écris les remerciements ». Cependant, je n'avais pas mesuré à quel point cela serait difficile, tant sont nombreuses les personnes que j'aimerais ici remercier, et que je crains d'oublier. Je pourrais évidemment me fendre d'un « merci à tous », mais je vais quand même me risquer à l'exercice.


Merci tout d'abord à ceux qui m'ont permis d'effectuer ces trois années de thèse. À commencer par Franck, pour ces presque 4 années de collaboration, pour ta gentillesse, ta disponibilité, bref pour toutes tes qualités humaines qui t'ont permis de me supporter ! Merci aussi à toi Wolfgang, pour tout ce que tu m'as appris en physique, pour les trails en montagne et cette ascension au pas de course de la croix de Belledone. Merci à vous deux, de m'avoir permis de réaliser ma thèse dans de si bonnes conditions.


Merci à tous ceux et celles qui font que l'institut fonctionne, qui s'occupent de nos missions, de nos petits problèmes (un merci particulier à Sabine et Véronique), à ceux qui réparent ce que les thésards cassent, ou bien encore ceux que les thésards brulent (n'est-ce pas Daniel et Éric) ! En particulier, un grand merci à Éric que je n'ai pas cessé de mettre à l'épreuve et qui a su ressouder les fuites et me faire de nouvelles bobines ! Merci, à vous tous et aux bons souvenirs durant les journées du laboratoire (mais chut, ça reste entre nous !).


Merci aussi aux chercheurs et maîtres de conférence de l'institut, Laurent Cagnon pour l'ALD, Serge Florens et Arnaud Ralko pour la théorie, et tous ceux avec qui j'ai pu échanger que ce soit sur des sujets scientifiques ou bien sur la vie de tous les jours. Une pensée particulière pour tous les participants aux soirées Nanars.


Merci également à ceux qui ont fait partie de mon équipe. Les thésards tout d'abord, à Nicolas Roch de m'avoir accueilli au sein de l'équipe durant mes quelques mois de stage, sans qui les travaux que je présente dans cette thèse n'auraient pas été possible. Merci également à Romain Maurant pour les nombreuses discussions scientifiques autour d'un tableau et ces moments de franche rigolade entre co-bureaux, merci aussi à Marc, Jarno, Ramanpez, Viet, Adrien, Shubba et Stefan qui a pris brillamment la relève. Je tiens également à remercier Edgar Bonet et Christophe Thirion pour le développement de nos moyens de mesure, et puis parce que lorsqu'on a une question, Edgar a toujours une réponse. Merci aussi à Vincent Bouchiat pour nos discussions et tes remarques toujours pertinentes, à Nedjma Bandiab et Laëtitia Marty pour votre bonne humeur.


Et comment ne pas parler de mes co-bureaux ! Un gros merci à Raoul Piquerel, Antoine Reserbat-Plantey et Matias Urdampilleta. Raoul, d'abord, pour les journées de grimpe, les apéros du soir et les fêtes de Dax (entre autres choses). Antoine, merci pour les bonnes bouffes, la musique Jazz et la dégustation de vin ! Quant à Matias (el basquoil), merci également pour toutes les bonnes soirées, les discussions sans fin sur le magnétisme moléculaire et sur bien d'autres sujets. Vous allez me manquer et grâce à vous, je ne monte plus les escaliers de la même façon !
Merci aux thésards et thésardes qui ont contribué à l'ambiance de l'institut : Orianne, Vars, Geta, Cristophe, et la liste est encore longue. Un grand merci à ceux également qui ont partagé quelques bons moments en Roumanie, en Italie (notamment la Sardaigne).


Merci aussi à ceux qui ont rendu la vie en dehors du laboratoire (eh oui, ça arrive) très agréable ; sur Grenoble tout d'abord avec les amis de mon aventure suisse et italienne (ils se reconnaitront), à Dri, Charlène, Sophie, Julie, la kika et ses maîtres et bien d'autres encore ; et désolé de ne jamais répondre aux textos dans les temps. Merci également aux Landais qui m'accueillent quand je rentre et qui ont, pour certains, couru sous le même maillot. Ils se reconnaitront !


Je remercie bien sûr ma famille qui m'a toujours soutenu, même quand mes projets étaient un peu fous ! Si j'ai réussi à les mener à terme, c'est avant tout à vous que je le dois. Enfin, je remercie celle qui m'accompagne, Christelle, merci de m'avoir supporté et d'avoir toujours été là quand l'ampleur de la tâche me semblait trop grande, cette thèse est aussi un peu la tienne, et je te la dédie.

\vspace{1cm}
\hspace{8cm} Romain VINCENT