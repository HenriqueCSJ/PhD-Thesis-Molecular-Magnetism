\chapter{Le post-traitement}

Les mesures que j'ai eu à effectuer m'ont mis face à un problème de taille des données à traiter. Pour mesurer par exemple la population des spins nucléaires, il fallait analyser 22 000 balayages en champ constitués de plus de 1000 points chacun. Cela passait d'abord par l'application d'un filtre. Le signal en résultant devait être ensuite filtré afin d'identifier le minimum et le maximum. Cette opération devait être répété pour chacun des balayages. J'ai donc décidé de ne pas utiliser les logiciels habituels type MatLab ou Igor mais de me tourner vers un langage de programmation à proprement parler. Cette solution se justifie par deux paramètres. Tout d'abord, choisir un langage nous assure une indépendance du système d'exploitation utilisé. Deplus, choisir un langage de programmation permet de bénificier d'une large communauté de programmeur succeptible de venir en aide à chaque instant. De plus, cela permet de s'affranchir de l'achat d'une licence. Cette dernière considération est plus volontier philosophique que pratique dans notre cas mais elle permet de mette une partie ou la totalité du travail effectué à la disposition du reste de la communauté.

\section{Les type de langage}
Dans une classification simplifié, on peut identifié deux classes de langage : les langages compilés et les langages interprétés. Par exemple, le langage C appartient aux langages compilé. Cela signifie que le progamme contenue dans un fichier .c ne peut pas être directement utilisé mais doit être compilé afin de produire un fichier binaire qui lui pourra être utiliser. Les avantages et inconvénients de ce type de langage sera détaillé dans la suite. Le javascript très utiliser dans le monde internet est lui un langage interprété. Un fichier javascript est directement utiliser car il est ``lu'' et ``interprété'' directement par l'odinateur ou plus précisement par l'interpreteur. Les avantages de ce type de langage sera abordé dans la deuxième partie de cette section.

\subsection{Compilé}
\subsection{Interprété}

\section{Le bon compromis : Python}
\subsection{Python}
\subsection{Cython}
\subsection{Ipython}

\section{Méthode}
\subsection{Le filtre}
\subsection{Les histogrammes 2D}



