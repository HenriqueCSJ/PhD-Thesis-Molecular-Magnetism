\chapter*{Introduction}
\markboth{\MakeUppercase{Introduction}}{\MakeUppercase{Introduction}}
\addcontentsline{toc}{chapter}{Introduction}
\setcounter{figure}{0}

Les travaux que nous allons présenter se placent à la croisée de trois domaines : la spintronique, l'électronique organique et la magnétisme moléculaire. Le but est de créer un dispositif électronique mettant en jeu les propriétés magnétiques des molécules aimants. En effet ces trois domaines, pour répondre à des problématiques qui leur sont propres, et que nous détaillerons dans le premier chapitre, ont choisi de s'intéresser aux dispositifs de la spintronique moléculaire.  De tel dispositifs pourraient avoir des applications dans deux domaines distincts : l'électronique conventionnelle et l'information quantique.

Pour cela, il faut tout d'abord \^etre capable de fabriquer un dispositif qui préserve les propriétés magnétique des molécule aimant le constituant. Nous avons choisi d'adopter une configuration transistor dans laquelle le r\^ole de canal est joué par un aimant moléculaire. Pour cela, il nous faut tout d'abord être capable de ``piéger" une molécule entre deux électrodes. Ceci est rendu possible par l'utilisation du phénomène d’électromigration, bien connu des micro-électroniciens. Des échantillons permettant de mettre en œuvre cette technique efficacement ont du être développé, utilisant notamment la technique de dépôt par couche atomique~(ALD en anglais) pour obtenir une grille efficace. L'ensemble de la procédure de fabrication sera traitée dans le premier chapitre, en insistant en particulier la procédure d'électromigration et la fabrication d'une grille efficace.

