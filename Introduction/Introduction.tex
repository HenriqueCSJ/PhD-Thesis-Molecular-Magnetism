\chapter*{Introduction}
\markboth{Introduction}{Introduction}
\addcontentsline{toc}{chapter}{Introduction}
\setcounter{figure}{0}

\section*{Motivations}
L'électronique a pris une place centrale dans la vie de tous les jours. Elle est impliquée dans le moindre de nos gestes : du simple micro-ondes au super calculateur nous donnant la météo de la semaine. A tel point qu'elle offre aujourd'hui un angle d'attaque nouveau jusque dans l'analyse de la société~(cf concept de fracture numérique). Cette place centrale oblige les acteurs de cette industrie à relever deux défis majeurs : fabriquer des composants moins chers et plus performants. Pendant longtemps la pierre angulaire de cette course à la performance a été la réduction de la taille des composants jusqu'à atteindre, à l'heure actuelle, des tailles caractéristiques de quelques dizaines de nanomètres. Mais ce paradigme du toujours plus petit a atteint ses limites et la technique impliquée dans la fabrication des dispositifs devrait bientôt marquer le pas. Les performances se gagnent désormais par l'implémentation de nouvelles architectures comme le prouve l'apparition, depuis quelques années déjà, de processeurs multi-coeurs, ou bien encore, par l'amélioration des systèmes d'exploitation afin de tirer parti, de manière efficace, de ces nouvelles architectures. Mais en aucun cas, les solutions proposées ne font appel à de nouveaux concepts de la physique ou bien encore de la chimie. Or, c'est bien de ces deux disciplines que pourraient venir les prochaines évolutions majeures de l'électronique.

La chimie, tout d'abord, étudie depuis longtemps déjà les phénomènes d'auto-organisation. Ce terme traduit la tendance qu'ont certaines molécules à s'organiser de façon à former des structures complexes, sans intervention extérieure. Cette capacité à l'auto-organisation pourrait être utilisée dans l'avenir pour fabriquer, sans l'usage de techniques lithographiques, des composants micro-électroniques organiques. Ceci est d'autant plus vraisemblable que la chimie organique sait désormais synthétiser des composés conducteurs ou semi-conducteurs, déjà mis en œuvre dans des composants conventionnels tels que les transistors à films fins organiques. Mais l'électronique moléculaire, pour être une candidate sérieuse, doit également offrir la possibilité de fabriquer des composants de spintronique, c'est-à-dire, des composants sensibles au spin des électrons. A l'heure actuelle, ces composants sont au centre des technologies les plus courantes, tels que les disques durs ou bien encore les mémoires magnétiques~(MRAM en anglais). La chimie dispose pour cela de molécules possédant des propriétés magnétiques particulières : les aimants moléculaires. Ces derniers pourraient, par exemple, \^etre utilisés pour la fabrication d'électrodes ferromagnétiques organiques, ou bien encore, dans l'implémentation de vannes de spin.

Mais changer les méthodes de fabrication n'est qu'une partie de la solution. Il est également possible d'utiliser les nouveaux concepts issus de la mécanique quantique, comme le montre les résultats d'une discipline relativement récente : l'information quantique. Cette dernière a notamment des implications dans deux domaines de l'informatique moderne : la recherche dans les bases de données~(à l'aide de l’algorithme de Grover) et la cryptographie~(grâce à l'algorithme de Shor mais également de par les propriétés des observables en mécanique quantique). Les bases de données sont au centre de nombreuses applications et ont à faire face à un nombre toujours plus grand d'informations à traiter. Pour prendre l'exemple du moteur de recherche Google, celui-ci doit traiter les informations extraites de 30 trillions de documents, et ce, à raison de 40000 requêtes par seconde~(chiffres de Google en 2012). Et ces chiffres sont en constante hausse et nécessitent donc une puissance de calcul croissante. Une alternative à cette course à la puissance pourrait être trouvée dans la fabrication d'ordinateurs quantiques qui, grace à l'algorithme de Grover, pourraient réduire non seulement la puissance nécessaire mais également la mémoire utilisée. De m\^eme, face à la croissance des flux de données, mais aussi de la cybercriminalité, il est devenu primordial de mettre en place des solutions de cryptage efficaces. C'est à cette t\^ache que se sont attelés les spécialistes de ce que l’on appelle la cryptographie quantique. Le concept de réduction du paquet d'onde, par exemple, pourrait jouer un rôle central dans la détection d'interception, par un tiers, d'un signal envoyé.

Pour rendre cette révolution possible, il faut pouvoir stocker et manipuler les données à l'aide de bits quantiques ou qbits. De nombreux objets physiques peuvent implémenter ces fameux qbits au sein des laboratoires, mais lorsque l'on songe à une application à plus grande échelle, la liste des candidats diminue de façon significative. Les aimants moléculaires sont des prétendants sérieux à la fabrication de cette nouvelle électronique. Comme nous le détaillerons dans la suite, ils peuvent faire l'objet d'une approche ``\textit{Bottom-Up}", c'est-à-dire, s'appuyant sur l'auto-organisation de la matière. En outre, ils possèdent un moment magnétique dont l'orientation permettrait de coder l'information binaire, comme c'est déjà le cas dans les disques durs par exemple. Enfin, les aimants moléculaires constituent des systèmes quantiques susceptibles d'\^etre manipulés et pourraient donc \^etre utilisés en tant que qbits.

Les travaux que nous allons présenter dans la suite tentent de répondre aux problématiques que nous venons d'évoquer en montrant qu'une spintronique moléculaire est possible. Cette dernière permettrait d'utiliser les molécules aimants comme des qbits en utilisant non seulement le moment magnétique électronique comme support de l'information, mais également le spin nucléaire. Cette dernière possibilité rend les aimants moléculaires d'autant plus attractifs que le spin nucléaire a été présenté comme composant de base idéal à l'informatique quantique~\cite{Kane1998}.

\newpage
 
\section*{Plan de thèse}

Nous commencerons, dans le premier chapitre, par dresser un rapide aperçu de la spintronique et de l'électronique organique, soulignant les problématiques inhérentes à  chacun de ces domaines. Nous discuterons ensuite du magnétisme moléculaire et des synergies possible avec l'électronique, aboutissant à la spintronique moléculaire. Nous continuerons par une brève histoire de l'électronique moléculaire avec la réalisation du premier transistor à molécule, puis nous présenterons les évolutions ayant conduit à la réalisation du premier dispositif de spintronique moléculaire à l'échelle d'une molécule unique. Enfin, nous retracerons les évolutions de cette thématique au sein de notre groupe pour conclure en évoquant les tous derniers résultats obtenus.

Dans une deuxième partie, nous décrirons les techniques de fabrication nous permettant d'obtenir un transistor à molécule unique. Nous insisterons tout d'abord sur la nécessité de disposer d'une bonne grille, en soulignant les critères importants et les techniques de fabrication permettant de les remplir. Nous détaillerons ensuite la fabrication des nanofils d'or ainsi que la technique d'électromigration permettant d'obtenir nos nano-cassures. Nous terminerons par une brève présentation de notre technique de dépôt de molécules et une description succincte de la première caractérisation électrique.

Enfin dans le dernier chapitre, nous exposerons l'ensemble de nos résultats expérimentaux. Nous détaillerons dans un premier temps ce qui fait du terbium un candidat idéal aux applications de spintronique moléculaire et nous décrirons ses principales propriétés magnétiques. Nous analyserons ensuite les différents mécanismes susceptibles de coupler le transport électronique au magnétisme moléculaire, en envisageant deux configurations : l'une directe et l'autre indirecte. Puis nous nous intéresserons aux propriétés électroniques de notre transistor. En particulier, nous extrairons de cette analyse l'intensité et la nature de l'interaction entre les électrons donnant naissance au courant mesuré et le moment magnétique moléculaire. Ensuite, nous analyserons plus finement les sauts de conductance différentielle associés au retournement de l'aimantation. Nous détaillerons tout d'abord notre procédure de détection et d'analyse des sauts de conductance. Nous montrerons qu'ils permettent, par une analyse de leur position en champ magnétique, de mesurer indirectement l’état du spin nucléaire du terbium. En outre, nous utiliserons l'hystérésis en conductance induite par le couplage magnéto-transport pour identifier les différents axes de notre molécule aimant. Puis nous nous intéresserons au magnétisme électronique de l'aimant moléculaire TbPc$_{2}$ en insistant sur le rôle central du champ transverse dans nos mesures. Enfin, nous analyserons en détail la dynamique du magnétisme nucléaire. En particulier, nous mettrons en évidence un temps de vie des états nucléaires de près de dix secondes, ainsi que l'aspect non-destructif de notre technique de mesure. Nous démontrerons également que cette dynamique est fortement influencée par l'environnement électrostatique de notre système. De plus, part une analyse des populations nucléaires nous mettrons en évidence la bonne thermalisation de ce dernier malgré le courant électrique traversant notre transistor.