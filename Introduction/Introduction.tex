\chapter*{Introduction}
\markboth{\MakeUppercase{Introduction}}{\MakeUppercase{Introduction}}
\addcontentsline{toc}{chapter}{Introduction}
\setcounter{figure}{0}

\section*{Motivation}
L'électronique à pris une place centrale dans la vie de tous les jours. Elle est impliquée dans le moindre de nos gestes : du simple micro-onde au super calculateur nous donnant la météo de la semaine. A tel point qu'elle offre aujourd'hui un angle d'attaque nouveau jusque dans l'analyse de la société. Ceci a donné lieu à des concepts tels que la fracture numérique qui recouvre la fracture géographique, liée à l'implantation inégale des réseaux d'information, et la fracture générationnelle qui traduit le manque de formation des générations de l'avant numérique. Cette place centrale oblige les acteurs de cette industrie à relever deux défis majeurs : fabriquer des composants moins chers et plus performants. Pendant longtemps la pierre angulaire de cette course à la performance à été la réduction de la taille des composants, jusqu'à atteindre, à l'heure actuelle, des tailles caractéristiques de quelques dizaines de nanomètre. Mais ce paradigme du toujours plus petit à atteint ces limites et la technique impliquée dans la fabrication des dispositifs devrait bientôt marquer le pas. Les performances se gagnent désormais par l'implémentation de nouvelles architectures comme le prouve l'apparition, depuis quelques années déjà, de processeurs multi-coeurs, ou bien encore, par l'amélioration des systèmes d'exploitation afin de tirer parti, de manières efficace, de ces nouvelles architectures. Mais en aucun cas les solutions proposées ne font appel à de nouveaux concepts de la physique ou bien encore de la chimie. Or, c'est bien de ces deux disciplines que pourrait venir les prochaines évolutions majeures de l'électronique.

La chimie, tout d'abord, étudie depuis longtemps déjà les phénomènes d'auto-organisation. Ce terme traduit la tendance qu'on certaine molécule de s'organiser de façon à former des structures complexes, sans intervention extérieure. Cette capacité à l'auto-organisation pourrait être utilisée dans l'avenir pour fabriquer, sans l'usage de techniques lithographiques, des composants micro-électronique organique. Ceci est d'autant plus vraisemblable que la chimie organique sait désormais synthétiser des composés conducteurs ou semi-conducteurs, déjà mis en œuvre dans des composants conventionnels tel que les transistors à films fins organique. Mais l'électronique organique, pour être une candidate sérieuse, doit également offrir la possibilité de fabriquer des composants de spintronique, c'est-à-dire, des composants sensibles au spin des électrons. A l'heure actuelle, c'est composant sont au centre des technologies les plus courantes comme les disques durs ou bien encore les mémoires magnétiques~(MRAM en anglais). Elle dispose pour cela de molécule possédant des propriétés magnétiques particulières : les aimants moléculaires. Ces derniers pourrait par exemple \^etre utilisés pour la fabrication d'électrodes ferromagnétiques, ou bien encore, dans l'implémentation de vannes de spin.

Mais changer les méthodes de fabrication n'est qu'une partie de la solution. Il est également possible d'utiliser les nouveaux concepts issus de la mécanique quantique et synthétisés dans une discipline relativement récente : l'information quantique. Cette dernière a notamment des implications dans deux domaines de l'informatique moderne : la recherche dans les bases de données~(à l'aide de l’algorithme de Grover) et la cryptographie~(notamment grâce à l'algorithme de Shor). Les bases de données sont au centre de nombreuses applications et ont a faire face à un nombre toujours plus grand d'informations à traiter. Pour prendre l'exemple de Google, le moteur de recherche doit traiter les informations extraites de 30 trillions de documents, et ce, à raison de 40000 requêtes par seconde~(chiffres de Google en 2012). Et ces chiffres sont en constante hausse et nécessitent donc une puissance de calcul croissante. Une alternative à cette course à la puissance pourrait être trouvée dans la fabrication d'ordinateurs quantiques qui, grace à l'algorithme de Grover, pourraient réduire la puissance mais également la mémoire nécessaire. De m\^eme, face à la croissance des flux de donnée, mais aussi de la cybercriminalité, il est devenu primordial de mettre en place des solutions de cryptage efficace. C'est à cette tache que se sont attelé les spécialiste de ce que l’on appelle la cryptographie quantique.

Une des briques essentielles à cette révolution est  ce que l'on appelle le bit quantique ou qbit. De nombreux objets physiques peuvent implémenter ces fameux qbits au sein des laboratoires, mais lorsque l'on songe à une application à plus grande échelle, la liste des candidats diminue de façon significative. Les aimants moléculaires constituent des candidats sérieux dans la fabrication de cette nouvelle électronique. Comme nous le détaillerons dans la suite, ils peuvent faire l'objet d'une approche ``\textit{Bottom-Up}", c'est-à-dire, s'appuyant sur l'auto-organisation de la matière. En outre, ils possèdent un moment magnétique dont l'orientation permettrait de coder l'information binaire, comme c'est déjà le cas dans les disques durs par exemple. Enfin, les aimants moléculaires constituent des systèmes quantiques susceptibles d'\^etre manipulés et pourraient donc \^etre utilisés en tant que qbits.


Les travaux que nous allons présenter dans la suite tentent de répondre aux problématiques que nous venons d'évoquer en montrant qu'une spintronique moléculaire est possible. Cette dernière permettrait d'utiliser les molécules aimants comme des qbits en utilisant non seulement le moment magnétique électronique comme vecteur de l'information, mais également le spin nucléaire. Cette dernière possibilité rend les aimants moléculaire d'autant plus attractifs que le spin nucléaire a été présenté comme composant de base idéal à l'informatique quantique.

\section*{Plan de thèse}

Nous commencerons, dans le premier chapitre, par dresser une rapide aperçu de la spintronique et de l'électronique organique, soulignant les problématiques inhérentes à  chacun de ces domaines. Nous discuterons ensuite du magnétisme moléculaire et des synergies possible avec l'électronique aboutissant à la spintronique moléculaire. Nous continuerons par une brève histoire de l'électronique moléculaire avec la réalisation du premier transistor à molécule, puis nous présenterons les évolutions aillant conduit à la réalisation du premier dispositif de spintronique moléculaire à l'échelle d'une molécule unique. Enfin, nous retracerons un peu les évolutions de cette thématique au sein de notre groupe pour conclure par les tout derniers résultats obtenus.

Dans une deuxième partie, nous décrirons les techniques de fabrication nous permettant d'obtenir un transistor à molécule unique. Nous insisterons tout d'abord sur la nécessité de disposer d'une bonne grille en soulignant les critères importants et les techniques de fabrication permettant de les remplir. Nous détaillerons ensuite la fabrication des nanofils d'or ainsi que la technique d'électromigration permettant d'obtenir nos nano-cassures. Nous terminerons par une brève présentation de notre technique de dépôt de molécule et une description succincte de la première caractérisation électrique.

